\documentclass[12pt]{scrartcl}

\usepackage[
  a4paper, mag=1000,
  left=2cm, right=1cm, top=2cm, bottom=2cm, headsep=0.7cm, footskip=1.27cm
]{geometry}

\usepackage[T2A]{fontenc}
\usepackage[utf8]{inputenc}
\usepackage[english,russian]{babel}
\usepackage{cmap}
\usepackage{amsmath}
\usepackage{tabularx}
\usepackage{array}
\usepackage{graphicx}
\IfFileExists{pscyr.sty}{\usepackage{pscyr}}{}
\usepackage[parfill]{parskip}
\usepackage{lastpage}
\usepackage{setspace} % single spacing between lines
\usepackage{blindtext} % for generated text - can remove
\usepackage{titlesec} % set header spacing
\setlength{\parindent}{15pt} % paragraph indent

\titlespacing{\section}{0pt}{\parskip}{-\parskip}
\titlespacing{\subsection}{0pt}{\parskip}{-\parskip}
\titlespacing{\subsubsection}{0pt}{\parskip}{-\parskip}

\usepackage[numbered]{bookmark}
\clubpenalty=10000
\widowpenalty=10000

\usepackage{fancybox,fancyhdr}
\pagestyle{fancy}
\fancyhf{}
\fancyhead[C]{\small{Олимпиадное программирование (средний уровень). Перебор --- день 03. \\ Летняя компьютерная школа ``КЭШ'', 8--28 августа 2017 года}}

%user-defined commands

\newcommand{\inputFile}{стандартный ввод}
\newcommand{\outputFile}{стандартный вывод}

\begin{document}

\singlespacing

\section*{Задача A. Пифагоровы тройки}

\begin{tabularx}{\textwidth}{l l X}
    Имя входного файла: & \texttt{\inputFile} \\
    Имя выходного файла: & \texttt{\outputFile} \\
    Ограничение по времени: & $2$ секунды \\
    Ограничение по памяти: & $256$ мегабайт \\
\end{tabularx}

$a^2+b^2=c^2$ - совокупность чисел a, b и с называется пифагоровой тройкой. 
Найдите и выведите все пифагоровы тройки, у которых числа a, b, и с меньше введеного числа N. 

\subsection*{Формат входных данных}
Число $N$, где $N \leq 100$

\subsection*{Формат выходных данных}
Все найденные пифагоровы тройки, при этом числа a, b и с выводятся в одну строку через пробел, а каждая новая тройка с новой строки. 


\texttt{
    \begin{tabularx}{0.9\textwidth}{| X | X |}
        \hline
        \multicolumn{1}{|c|}{Ввод} & \multicolumn{1}{c|}{Вывод} \\ \hline
        \parbox[t]{\textheight}{
           20 \\
        } & \parbox[t]{\textheight}{
            % Пример вывода. Каждая строчка заканчивается на \\ 
            3 4 5 \\
            5 12 13 \\
            6 8 10 \\
            8 15 17 \\
            9 12 15 \\
            12 16 20 \\
        } \\ \hline
    \end{tabularx}
}
\newpage



\section*{Задача B. Задача о рюкзаке}

\begin{tabularx}{\textwidth}{l l X}
    Имя входного файла: & \texttt{\inputFile} \\
    Имя выходного файла: & \texttt{\outputFile} \\
    Ограничение по времени: & $2$ секунды \\
    Ограничение по памяти: & $256$ мегабайт \\
\end{tabularx}

Пусть имеется набор предметов, каждый из которых имеет два параметра — вес и ценность. Также имеется рюкзак определённой вместимости. 
Задача заключается в том, чтобы собрать рюкзак с максимальной ценностью предметов внутри, соблюдая при этом ограничение рюкзака на суммарный вес.

\subsection*{Формат входных данных}
В первой строке через пробел перечислены общее предметов, которые можно класть в рюкзак ($3 \leq N \leq 20$) и максимальный вес ранца ($3 \leq M \leq 600$) в килограммах.
В следующих $N$ строчках через пробел перечислены стоимость ($N_i$) и вес ($M_i$) отдельно взятого предмета.

\subsection*{Формат выходных данных}
Вывести одно единственное число --- максимальную стоимость предмета без перегруза рюкзака.

\subsection*{Примеры}

\texttt{
    \begin{tabularx}{0.9\textwidth}{| X | X |}
        \hline
        \multicolumn{1}{|c|}{Ввод} & \multicolumn{1}{c|}{Вывод} \\ \hline
        \parbox[t]{\textheight}{
            5 13 \\
            1 3 \\
            6 4 \\
            4 5 \\
            7 8 \\
            6 9 \\
        } & \parbox[t]{\textheight}{
            % Пример вывода. Каждая строчка заканчивается на \\ 
            13 \\
        } \\ \hline
    \end{tabularx}
}

\newpage

\section*{Задача С. Пифагоровы тройки x2}

\begin{tabularx}{\textwidth}{l l X}
    Имя входного файла: & \texttt{\inputFile} \\
    Имя выходного файла: & \texttt{\outputFile} \\
    Ограничение по времени: & $2$ секунды \\
    Ограничение по памяти: & $256$ мегабайт \\
\end{tabularx}

$a^2+b^2=c^2$ - совокупность чисел a, b и с называется пифагоровой тройкой. 
Найдите и выведите все пифагоровы тройки, у которых числа a, b, и с меньше введеного числа N. 

\subsection*{Формат входных данных}
Число $N$, где $N \leq 1000$

\subsection*{Формат выходных данных}
Все найденные пифагоровы тройки, при этом числа a, b и с выводятся в одну строку через пробел, а каждая новая тройка с новой строки. 


\texttt{
    \begin{tabularx}{0.9\textwidth}{| X | X |}
        \hline
        \multicolumn{1}{|c|}{Ввод} & \multicolumn{1}{c|}{Вывод} \\ \hline
        \parbox[t]{\textheight}{
           20 \\
        } & \parbox[t]{\textheight}{
            % Пример вывода. Каждая строчка заканчивается на \\ 
            3 4 5 \\
            5 12 13 \\
            6 8 10 \\
            8 15 17 \\
            9 12 15 \\
            12 16 20 \\
        } \\ \hline
    \end{tabularx}
}
\newpage





\end{document}