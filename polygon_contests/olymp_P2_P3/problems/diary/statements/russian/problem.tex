\begin{problem}{Дневник заданий}{стандартный ввод}{стандартный вывод}{1 секунда}{256 мегабайт}

\begin{figure}[h]
\hspace*{\fill}
\includegraphics[width=\linewidth,natwidth=800,natheight=691]{B.jpg}
\hspace*{\fill}
\end{figure}

Геральт из Ривии будучи почетным ведьмаком, чтобы не забыть ни один из своих многочисленных заказов, ведет дневник. 
В нем он каждый день записывает заказы, каждый из которых имеет номер (индивидуальный для каждого дня). 

После удачно выполненного задания, а именно избавления этого бренного мира от нового монстра путем лишения того головы и других не менее приятных методов, 
он ставит перед номером заказа знак минус, чтобы знать, что можно отправляться за заслуженной наградой. 
Однажды, сидя в кабаке со своими друзьями Золтаном и Лютиком, Геральт поспорил с ними на то, что он каждый день убивает больше 5 монстров. Друзья ему не поверили и 
Геральт достал свою заветную книжонку. Помогите друзьям проверить, кто же прав.

\InputFile
На вход подается число $N (0 < N < 100)$~---~количество дней, за которые друзья просматривают записи заказов. 
Далее будут описаны N дней, то есть будет введено число $M (0 < M < 10)$~---~количество заказов в $i$-й день, 
и описание самих заказов, каждый номер заказа по модулю не певосходит $100$. (номер может быть отрицательным, ведь мы все еще помним, как Геральт отмечает выполненный задания)

\OutputFile
Для каждого дня выяснить, сколько заказов было успешно выполнено. Вывести полученные значения через пробел.

\Example

\begin{example}
\exmp{3
3
1 -2 3
2
1 2
4
-2 3 -4 -5
}{1 0 3 
}%
\end{example}

\end{problem}

