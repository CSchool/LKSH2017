\begin{problem}{Сортировка Хакуна Матата}{стандартный ввод}{стандартный вывод}{1 секунда}{256 мегабайт}

После веселой прогулки по африканской долине Тимон и Пумба решили пообедать. 
Для этого они достали из-под камня $n$ жуков, которые имеют 
различные целые массы от $1$ до $n$ и расставили их на ближайшем
бревне. Чтобы обед прошел веселее, Пумба предложил есть жуков по возрастанию масс. 

Чтобы отсортировать жуков по возрастанию масс, Тимон и Пумба используют сортировку Хакуна Матата.
Суть сортировки заключается в том, что Тимон и Пумба выбирают жуков, отличающихся массой не более чем на единицу, 
после чего друзья меняют их местами. Например, если Пумба выбрал жука с массой 2,
тогда Тимон может взять жука либо с массой 1, либо с массой 3. Так как друзья много раз пользовались 
этой сортировкой, им известно, что она всегда работает.
 
За всем этим процессом внимательно наблюдал Зазу. Он впервые видит нечто подобное, и ему интересно, как 
таким способом можно отсортировать жуков. Ваша задача~--- написать программу, которая по изначальному расположению жуков
выведет последовательность ходов для сортировки жуков по возрастанию масс.

\InputFile
В самой первой строке написано число $n$~--- количество жуков($1 \le n \le 100$).
Во второй строке заданы $n$ разделенных пробелами различных чисел $m_i$ ($1 \le m_i \le n$), которые обозначают массу жука 
с номером $i$.

\OutputFile
Выведите на первой строчке выходного файла число $t$~--- количество ходов в возможной сортировке Хакуна Матата.
На следующих $t$ строчках выведите последовательность ходов сортировки Хакуна Матата таким образом, чтобы на строчке 
с номером $s$ было выведено два числа $i$ и $j$, что значит, что на шаге с номером $s$ Пумба взял жука 
стоящего на позиции $i$, а Тимон на позиции $j$. Выведите любой способ, количество операций в котором не превышает 50000.


\Example

\begin{example}
\exmp{3
2 3 1
}{2
1 3
2 3
}%
\end{example}

\end{problem}

