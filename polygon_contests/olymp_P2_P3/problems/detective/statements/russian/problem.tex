\begin{problem}{Любителям детективов}{стандартный ввод}{стандартный вывод}{1 секунда}{256 мегабайт}

В городском управлении милиции одного прибрежного города ведется расследование крупного дела, в котором могут быть замешаны сотрудники милиции. Было принято решение о тайной установке оборудования для просмотра информации, поступающей через Интернет. Под подозрение попадают два отдела, но добиться выделения денег на покупку двух комплектов оборудования не удалось. К счастью, внутренняя сеть управления имеет древовидную структуру, то есть каждый отдел имеет выход в Интернет через какой-либо другой отдел. Исключение составляет отдел по борьбе с компьютерными преступлениями, который имеет непосредственный доступ в Интернет по модемной линии. 

Можно было бы установить оборудование для слежения прямо в этом отделе, но для предотвращения злоупотреблений лучше найти такое расположение, чтобы нарушалась секретность как можно меньшего количества лишних отделов. 

Как наиболее опытному в подобных вопросах сотруднику, решение этой задачи поручили вам. Подчиненные уже пронумеровали все отделы числами натуральными числами, начиная с 1, первый номер присвоен отделу по борьбе с компьютерными преступлениями. 

\InputFile
Первая строка входного файла содержит натуральное число $n$ ($n \le 30000$) --- количество отделов. Во второй строке записаны номера отделов, за которыми необходимо установить слежение. На третьей строке находятся $n - 1$ натуральных чисел, $i$-е из них задает номер отдела, к которому подсоединен отдел $i + 1$. 

\OutputFile
В выходной файл выведите одно число --- номер отдела, в котором следует установить следящее оборудование. 

\Example

\begin{example}
\exmp{4
3 4
1 1 3
}{3
}%
\end{example}

\end{problem}

