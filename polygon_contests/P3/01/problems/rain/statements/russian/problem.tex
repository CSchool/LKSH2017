\begin{problem}{Цветной дождь}{стандартный ввод}{стандартный вывод}{1 секунда}{256 мегабайт}

В Банановой республике очень много холмов, соединенных мостами. На химическом заводе ``Валентин'' произошла авария, в результате чего испарилось экспериментальное удобрение. На следующий день выпал цветной дождь, причем он прошел только над холмами. В некоторых местах падали красные капли, в некоторых --- синие, а в остальных --- зелёные, в результате чего холмы стали соответствующего цвета. Президенту Банановой республики это понравилось, но ему захотелось покрасить мосты между вершинами холмов так, что мосты были покрашены в цвет холмов, которые они соединяют. К сожалению, если холмы разного цвета, то покрасить мост таким образом не удастся. Посчитайте таких ``плохих'' мостов.

\InputFile
В первой строке записаны два числа $N$ и $M$ --- количество холмов и мостов соответственно ($1 \le N \le 100$, $0 \le M \le \frac{N(N - 1)}{2}$). В следующих $M$ строках записаны по два числа $a$ и $b$ ($1 \le a, b \le N$), которые означают, что холмы $a$ и $b$ соединены мостом.
В последней строке записаны $N$ чисел $k_1, k_2 \ldots k_N$, которые обозначают цвет соответствующего холма: 1 --- красный, 2 --- синий, 3 --- жёлтый.

\OutputFile
Выведите количество мостов, соединяющих холмы разных цветов.

\Examples

\begin{example}
\exmp{1 0
1
}{0}%
\exmp{7 7
1 2
1 6
1 7
2 3
3 5
3 6
5 6
1 1 1 1 1 3 3
}{4}%
\end{example}

\end{problem}

