\begin{problem}{Хождение за золотом}{стандартный ввод}{стандартный вывод}{1 секунда}{256 мегабайт}

Однажды царь решил вознаградить одного из своих мудрецов за хорошую работу. Он привел его в прямоугольную комнату размером $N \times M$, в каждой клетке которой лежало несколько килограммов золота. Царь разрешил мудрецу обойти несколько клеток (переходя с клетки, где сейчас находится мудрец, в одну из четырех с ней соседних), и собрать все золото, которое попадется на его пути.

Вам дан маршрут мудреца. Требуется определить, сколько килограммов золота он собрал.

\InputFile
Во входном файле записан план комнаты. Сначала записано количество строк $N$, затем --- количество столбцов $M$ ($1 \le N \le 20$, $1 \le M \le 20$). Затем записано $N$ строк по $M$ чисел в каждой --- количество килограммов золота, которое лежит в данной клетке (число от 0 до 50). Далее записано число $X$ --- сколько клеток обошел мудрец. Далее записаны координаты этих клеток (координаты клетки - это два числа: первое определяет номер строки, второе - номер столбца, верхняя левая клетка на плане имеет координаты $(1,1)$, правая нижняя - $(N, M)$). Гарантируется, что мудрец не проходил по одной и той же клетке дважды. 

\OutputFile
В выходной файл выведите количество килограммов золота, которое собрал мудрец.

\Example

\begin{example}
\exmp{3 4
1 2 3 4
5 6 7 8
9 10 11 12
5
1 1
2 1
2 2
2 3
1 3
}{22
}%
\end{example}

\end{problem}

