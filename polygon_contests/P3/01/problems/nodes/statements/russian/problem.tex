\begin{problem}{Транспортные узлы}{стандартный ввод}{стандартный вывод}{1 секунда}{256 мегабайт}

В стране $N$-мерике $n$ городов. Некоторые из них соединены двухсторонними дорогами --- всего в стране $m$ дорог. Из некоторых городов выходит одна дорога, а некоторые являются настоящими транспортными узлами - из них выходит достаточно много дорог. В этой задаче будем называть город транспортным узлом, если из него выходит хотя бы $k$ дорог. 

Задано описание дорожной сети $N$-мерики. Необходимо найти все ее транспортные узлы. 

\InputFile
Первая строка входного файла содержит число $n$ городов ($1 \le n \le 10^4$) и число $m$ дорог ($0 \le m \le 10^5$). Каждая из последующих m строк описывает одну дорогу и содержит два числа $u$ и $v$ ($1 \le u, v \le n, u \neq v$) - номера городов, соединенных дорогами. Последняя строка входного файла содержит целое число $k$ ($1 \le k \le 10^4$).
Каждая дорога упоминается во входном файле не более одного раза. 


\OutputFile
В первой строке выведите число транспортных узлов. Во второй строке выведите их номера в порядке возрастания.

\Examples

\begin{example}
\exmp{2 1
1 2
1
}{2
1 2 
}%
\exmp{4 3
1 2
1 3
1 4
3
}{1
1 
}%
\end{example}

\end{problem}

