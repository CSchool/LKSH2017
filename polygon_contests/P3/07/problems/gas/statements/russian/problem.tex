\begin{problem}{Заправки}{стандартный ввод}{стандартный вывод}{1 секунда}{256 мегабайт}

В стране $N$ городов, некоторые из которых соединены между собой дорогами. Для того, чтобы проехать по одной дороге требуется один бак бензина. В каждом городе бак бензина имеет разную стоимость. Вам требуется добраться из первого города в $N$-й, потратив как можно меньшее количество денег. 

\InputFile
Во входном файле записано сначала число $N$ ($1 \le N \le 100$), затем идет $N$ чисел, $i$-е из которых задает стоимость бензина в $i$-м городе (все это целые числа из диапазона от 0 до 100). Затем идет число $M$ --- количество дорог в стране, далее идет описание самих дорог. Каждая дорога задается двумя числами - номерами городов, которые она соединяет. Все дороги двухсторонние (то есть по ним можно ездить как в одну, так и в другую сторону), между двумя городами всегда существует не более одной дороги, не существует дорог, ведущих из города в себя.

\OutputFile
В выходной файл выведите одно число --- суммарную стоимость маршрута или \t{-1}, если добраться невозможно. 

\Examples

\begin{example}
\exmp{4
1 10 2 15
4
1 2
1 3
4 2
4 3
}{3
}%
\exmp{4
1 10 2 15
0
}{-1
}%
\end{example}

\end{problem}

