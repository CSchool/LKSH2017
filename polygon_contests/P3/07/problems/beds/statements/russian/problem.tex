\begin{problem}{Грядки}{стандартный ввод}{стандартный вывод}{1 секунда}{256 мегабайт}

Прямоугольный садовый участок шириной $N$ и длиной $M$ метров разбит на квадраты со стороной 1 метр. На этом участке вскопаны грядки. Грядкой называется совокупность квадратов, удовлетворяющая таким условиям:

\begin{itemize}
\item из любого квадрата этой грядки можно попасть в любой другой квадрат этой же грядки, последовательно переходя по грядке из квадрата в квадрат через их общую сторону
\item никакие две грядки не пересекаются и не касаются друг друга ни по вертикальной, ни по горизонтальной сторонам квадратов (касание грядок углами квадратов допускается)
\end{itemize}

Подсчитайте количество грядок на садовом участке.

\InputFile
В первой строке находятся числа $N$ и $M$ через пробел ($1 \le N, M \le 200$), далее идут $N$ строк по $M$ символов. Символ \t{\#} обозначает территорию грядки, точка соответствует незанятой территории. Других символов в исходном файле нет.

\OutputFile
Вывести одно число --- количество грядок на садовом участке.

\Example

\begin{example}
\exmp{5 10
\#{}\#{}......\#{}.
.\#{}..\#{}...\#{}.
.\#{}\#{}\#{}....\#{}.
..\#{}\#{}....\#{}.
........\#{}.}{3}%
\end{example}

\end{problem}

