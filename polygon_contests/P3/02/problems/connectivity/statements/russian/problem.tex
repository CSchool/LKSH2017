\begin{problem}{Компоненты связности}{стандартный ввод}{стандартный вывод}{1 секунда}{256 мегабайт}

В неориентированном графе посчитать количество компонент связности. В графе могут быть петли и кратные ребра. 

\InputFile
Во входном файле записаны сначала два числа $N$ и $M$, задающие соответственно количество вершин и количество ребер ($1 \le N \le 100$, $0 \le M \le 10^4$), а затем перечисляются ребра. Каждое ребро задается номерами вершин, которые оно соединяет. 

\OutputFile
В выходной файл выведите одно число - количество компонент связности. 

\Examples

\begin{example}
\exmp{3 4
1 1
1 2
1 3
2 3
}{1
}%
\exmp{5 3
1 1
1 2
2 1
}{4
}%
\exmp{5 0
}{5
}%
\end{example}

\Note
Компонента связности графа --- некоторое множество вершин графа такое, что для любых двух вершин из этого множества существует путь из одной в другую, и не существует пути из вершины этого множества в вершину не из этого множества. 

Петля --- ребро, начало и конец которого находятся в одной и той же вершине.

Кратные рёбра --- несколько рёбер, связывающих одну и ту же пару вершин. 

\end{problem}

