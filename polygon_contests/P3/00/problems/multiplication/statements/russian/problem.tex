\begin{problem}{Таблица умножения}{стандартный ввод}{стандартный вывод}{1 секунда}{256 мегабайт}

Большой любитель математики Вова решил повесить у себя в комнате таблицу умножения. После некоторых раздумий он обнаружил, что обычная таблица умножения 10 на 10 уже не популярна в наши дни. Он решил повесить у себя в комнате таблицу n на m. Представив себе эту таблицу, Вова задался вопросом - сколько раз в ней встречается каждая из цифр от 0 до 9?


И прежде чем нарисовать эту таблицу Вова попросил вас написать программу, которая даст ответ на его вопрос. 
Как известно, в таблице умножения на пересечении строки $i$ и столбца $j$ записано число $i \cdot j$. 

\InputFile
Входной файл состоит из единственной строки, на которой через пробел записаны два натуральный числа $n$ и $m$, $1 \le n, m \le 1000$

\OutputFile
Выходной файл должен состоять из десяти строк. На строке $i$ выведите количество раз, которое Вове придется нарисовать цифру $i - 1$.

\Example

\begin{example}
\exmp{10 10
}{28
24
27
15
23
15
17
8
15
6
}%
\end{example}

\end{problem}

