\begin{problem}{Шашечная доска}{стандартный ввод}{стандартный вывод}{1 секунда}{256 мегабайт}

В каждой клетке шахматной доски $8 \times 8$ в произвольном порядке находится шашка одного из цветов: белая, чёрная, красная или зелёная.

\begin{tabular}{| l | l | l | l | l | l | l | l |}
    \hline
    2 & 1 & 1 & 0 & 3 & 0 & 3 & 1 \\ \hline
    0 & 1 & 1 & 3 & 3 & 1 & 0 & 0 \\ \hline
    1 & 2 & 1 & 3 & 1 & 0 & 1 & 2 \\ \hline
    1 & 1 & 1 & 1 & 2 & 2 & 1 & 0 \\ \hline
    1 & 1 & 1 & 0 & 1 & 2 & 1 & 2 \\ \hline
    0 & 1 & 0 & 1 & 1 & 2 & 1 & 1 \\ \hline
    0 & 0 & 0 & 0 & 0 & 0 & 0 & 0 \\ \hline
    1 & 1 & 1 & 2 & 2 & 2 & 3 & 3 \\ \hline
\end{tabular}

0 --- цвет и местоположение ЧЁРНОЙ шашки \\
1 --- цвет и местоположение БЕЛОЙ шашки \\
2 --- цвет и местоположение КРАСНОЙ шашки \\
3 --- цвет и местоположение ЗЕЛЁНОЙ шашки \\

\InputFile
Данные о шашках записаны построчно и без пробелов в строке и между строками.

\OutputFile
Составить программу, подсчитывающую количество шашек каждого цвета и выводящую результат в виде:
\begin{itemize}
\item данных о местоположении красных шашек (в остальных местах вывести знак \texttt{.})
\item количестве черных, белых, красных и зеленых шашек через пробел
\end{itemize}
Если шашки какого-либо цвета отсутствуют на доске, то вывести в файл сообщение \texttt{BAD INPUT LIST}.

\Example

\begin{example}
\exmp{21103031
01133100
12131012
11112210
11101212
01011211
00000000
11122233
}{2.......
........
.2.....2
....22..
.....2.2
.....2..
........
...222..
18 28 11 7
}%
\end{example}

\end{problem}

