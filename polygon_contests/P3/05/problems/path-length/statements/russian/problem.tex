\begin{problem}{Длина пути}{стандартный ввод}{стандартный вывод}{1 секунда}{256 мегабайт}

В неориентированном графе требуется найти длину минимального пути между двумя вершинами. Гарантируется, что путь существует.

\InputFile
Во входном файле записано сначала число $N$ --- количество вершин в графе ($1 \le N \le 100$). Затем записана матрица смежности (0 обозначает отсутствие ребра, 1 --- наличие ребра). Затем записаны номера двух вершин --- начальной и конечной.

\OutputFile
В выходной файл выведите одно число - длину пути (количество ребер, которые нужно пройти).

\Example

\begin{example}
\exmp{5
0 1 0 0 1
1 0 1 0 0
0 1 0 0 0
0 0 0 0 0
1 0 0 0 0
3 5
}{3
}%
\end{example}

\end{problem}

