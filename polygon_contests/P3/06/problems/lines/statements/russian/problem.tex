\begin{problem}{Шарики}{стандартный ввод}{стандартный вывод}{1 секунда}{256 мегабайт}

В таблице из $N$ строк и $N$ столбцов некоторые клетки заняты шариками, другие свободны. Выбран шарик, который нужно переместить, и место, куда его нужно переместить. Выбранный шарик за один шаг перемещается в соседнюю по горизонтали или вертикали свободную клетку. Требуется выяснить, возможно ли переместить шарик из начальной клетки в заданную, и, если возможно, то найти путь из наименьшего количества шагов.

\InputFile
В первой строке находится число $N$ ($2 \le N \le 40$), в следующих $N$ строках --- по $N$ символов. Символом точки обозначена свободная клетка, латинской заглавной \t{O} --- шарик, \t{@} --- исходное положение шарика, который должен двигаться, латинской заглавной \t{X} --- конечное положение шарика.


\OutputFile
В первой строке выводится \t{Y}, если движение возможно, или \t{N}, если нет. Если движение возможно, далее следует $N$ строк по $N$ символов --- как и на вводе, но буква X, а также все точки по пути заменяются плюсами. 

\Examples

\begin{example}
\exmp{5
....X
.OOOO
.....
OOOO.
@....
}{Y
+++++
+OOOO
+++++
OOOO+
@++++
}%
\exmp{5
..X..
.....
OOOOO
.....
..@..
}{N
}%
\exmp{5
...X.
.....
O.OOO
.....
....@
}{Y
...+.
.+++.
O+OOO
.+...
.+++@
}%
\end{example}

\end{problem}

