\begin{problem}{Путь спелеолога}{стандартный ввод}{стандартный вывод}{1 секунда}{256 мегабайт}

Пещера представлена кубом, разбитым на $N$ частей по каждому измерению (то есть на $N^3$ кубических клеток). Каждая клетка может быть или пустой, или полностью заполненной камнем. Исходя из положения спелеолога в пещере, требуется найти, какое минимальное количество перемещений по клеткам ему требуется, чтобы выбраться на поверхность. Переходить из клетки в клетку можно, только если они обе свободны и имеют общую грань.

\InputFile
В первой строке содержится число $N$ ($1 \le N \le 30$). Далее следует $N$ блоков. Блок состоит из пустой строки и $N$ строк по $N$ символов: \t{\#} --- обозначает клетку, заполненную камнями, точка --- свободную клетку. Начальное положение спелеолога обозначено заглавной буквой \t{S}. Первый блок представляет верхний уровень пещеры, достижение любой свободной его клетки означает выход на поверхность. Выход на поверхность всегда возможен.

\OutputFile
Вывести одно число - длину пути до поверхности.

\Example

\begin{example}
\exmp{3
\hfill \break
\#{}\#{}\#{}
\#{}\#{}\#{}
.\#{}\#{}
\hfill \break
.\#{}.
.\#{}S
.\#{}.
\hfill \break
\#{}\#{}\#{}
...
\#{}\#{}\#{}
}{6}%
\end{example}

\Note
В тесте из примера, нужно:
\begin{itemize}
\item спуститься на уровень вниз
\item сделать два движения на запад
\item подняться на уровень вверх
\item сделать движение на юг
\item подняться на уровень вверх
\end{itemize}

\end{problem}

