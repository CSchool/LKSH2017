\begin{problem}{Путь коня}{стандартный ввод}{стандартный вывод}{1 секунда}{256 мегабайт}

Дана шахматная доска, состоящая из $N \times N$ клеток, несколько из них вырезано. Провести ходом коня через невырезанные клетки путь минимальной длины из одной заданной клетки в другую.

\InputFile
В первой строке задано число $N$ ($2 \le N \le 50$). В следующих $N$ строках содержится по $N$ символов. Символом \t{\#} обозначена вырезанная клетка, точкой --- невырезанная клетка, \t{@} --- заданные клетки (таких символов два).

\OutputFile
Если путь построить невозможно, вывести \t{Impossible}, в противном случае вывести такую же карту, как и на входе, но пометить все промежуточные положения коня символом \t{@}.

\Examples

\begin{example}
\exmp{5
.....
.@@..
.....
.....
.....
}{.....
.@@..
...@.
.@...
.....
}%
\exmp{5
@..@.
..\#{}\#{}.
.....
.....
.....}{@..@.
..\#{}\#{}.
.@..@
..@..
@....}%
\exmp{5
@....
..\#{}..
.\#{}...
.....
....@}{Impossible}%
\end{example}

\end{problem}

