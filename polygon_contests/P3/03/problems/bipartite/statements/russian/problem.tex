\begin{problem}{Двудольный граф}{стандартный ввод}{стандартный вывод}{1 секунда}{256 мегабайт}

Граф называется двудольным, если его вершины можно раскрасить в два цвета так, что нет ребер, соединяющих вершины одинакового цвета (то есть каждое ребро идет из вершины 1-го цвета в вершину 2-го) 
Дан граф. Требуется проверить, является ли он двудольным, и если да, то раскрасить его вершины.

\InputFile
Во входном файле задано сначала число $N$ --- количество вершин графа (не превышает 100). Далее идет матрица смежности --- матрица размером $N \times N$ из 0 и 1 (0 обозначает отсутствие ребра, 1 --- наличие). Граф неориентированный, без петель.


\OutputFile
В первую строку выведите одно из сообщений \t{YES} или \t{NO} (в зависимости от того, является ли граф двудольным или нет). В случае ответа \t{YES} во второй строке выведите N чисел --- цвета, в которые нужно раскрасить вершины. 

\Examples

\begin{example}
\exmp{3
0 1 1
1 0 1
1 1 0
}{NO
}%
\exmp{3
0 1 1
1 0 0
1 0 0
}{YES
1 2 2 
}%
\end{example}

\end{problem}

