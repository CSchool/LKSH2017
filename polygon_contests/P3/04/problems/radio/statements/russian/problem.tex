\begin{problem}{Радиовышки}{стандартный ввод}{стандартный вывод}{1 секунда}{256 мегабайт}

В стране ``Флатландия'' появился новый оператор сотовой связи ``Flat Phone''. Этим оператором было построено $N$ радио вышек, каждая из которых может транслировать сигнал на одной из двух частот. На каждую из этих вышек требуется установить радио передатчик одной и той же мощности $R$, способный обеспечить уверенный прием и передачу сигнала на расстояние, равное $R$ километрам. Как известно, при распространении радио волн одной частоты возникает интерференция, ухудшающая качество сигнала. Поэтому оператором сотовой связи было принято решение выбрать для каждой вышки одну частоту работы радио передатчика таком образом, чтобы не существовало ни одного участка на поверхности, в котором бы перекрывались несколько зон уверенного приема и передачи сигнала равной частоты. При этом сотовый оператор хочет, чтобы мощность передатчиков была как можно больше.

Требуется написать программу, определяющую максимально возможное значение мощности передатчиков.

\InputFile
Первая строка содержит число $N$ ($3 \le N \le 100$) --- количество вышек. Последующие N строк содержат по два целых числа --- координаты вышек. Координаты заданы в километрах и не превышают $10^4$ по модулю. Все точки, в которых расположены вышки, различны.

\OutputFile
В первую и единственную строку выведите одно число --- искомое максимально возможное значение мощности передатчиков. Ответ должен быть записан с 5 знаками после десятичной точки.


\Example

\begin{example}
\exmp{4
0 0
1 0
1 1
0 1
}{0.7071067811865475
}%
\end{example}

\end{problem}

